\section{Electron}
\label{sec:electron}
Electron originally was released as Atom Shell by GitHub at \displaydate{dateAtomShell}~\cite{githubReleaseV010}.
The intention of the developers was ``to handle the Chromium/Node.js event loop integration and native APIs``~\cite{sawicki_2015} for the Atom Editor.
On \displaydate{dateRenameElectron} it was renamed to Electron and announced that the developers want to provide a framework that allows to build desktop applications with the using web-technologies.

\subsection{Architecture}
\label{subsec:electron:arch}
This subsection takes a deeper look at the underlying features and techniques like Chromium and so on.
\subsection{Frontend}
\label{subsec:electron:frontend}
This subsection deals with the front-end Electron is using, which other frontend-frameworks are supported, etc.
\subsection{Backend}
\label{subsec:electron:backend}
At this subsection we will take a look at the Node JS Backend of Electron and analyse the advantages and disadvantages of NodeJS