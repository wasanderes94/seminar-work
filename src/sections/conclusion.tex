\section{Conclusion}
\label{sec:conclusion}
%At this section a conclusion is made based on the advantages/disadvantages of each framework
%described in \ref{sec:conclusions}.
%Are there different use-cases for each framework, are they concurrent to each other, etc.?
Reminding the intention of the developers of Tauri mentioned at Chapter~\ref{sec:tauri} and contrasting them to the measurements discussed at Chapter~\ref{sec:implementation}
it can be determined that some aspects like execution time or security of Tauri applications compared to Electron are accurate.
Tauri applications have a more efficient performance in aspects of memory usage or execution time, due to the fact of using Rust as backend core.
Nevertheless, some benchmarks provided by Tauri, especially in case of memory consumption could not be reproduced or investigated further, since those are not provided public.
As mentioned in Chapter~\ref{sec:summary} Tauri, does a step towards the traditional web development compared to Electron.
This leads to the fact, that system-specific requirements have to be considered and therefore an application for each \ac{OS} has to be implemented at worst case.
Although this might need only small changes on the entire code base, redundant code is a consequence of this.
Tauri has experienced much attention since its first release, but at the current state is not suitable for big applications in production mode.
This statement is based on different aspects. \\ \
First, the documentation is difficult to understand resulting in paraphrases or one-sentenced explanations of entire architecture aspects like their design patterns, than actual explaining them.
This results in uncertainties of how the Tauri framework is solving tasks in detail.
Especially the security section is rather describing possible threats that can occur during development than explaining their implemented solutions or how they affect existing code.
Although documentations do not have to be objective, the Tauri documentation often compares parts of it against Electron applications even if some of them are referencing old and deprecated modules like ASAR files~\cite{tauri}. \\
Second, knowledge of Rust is required to implement more complex applications that need access to the Rust core.
Electron can be written by using only JavaScript and thus lowers the bounds for small web-development teams, since JavaScript is widely used at this branch.
For those teams, implementing a Tauri application will result in learning a complete new programming language.
In worst case they have to learn complete new programming patters like \ac{OOP} principles or memory management, if they do not have knowledge in comparable languages like C++.
This consumes more resources during the development process and thus will lead to an increasing timespan between the initial development and a release of the application.
Although Tauri has claimed to face this issue and announced to provide a backend that can be written using different languages, it is neither implemented nor scheduled yet~\cite{tauri}.
Third is the community itself as well as the support of the developers, since Tauri and especially Rust suffers from a lack of functionalities compared to Electron.
Several standard web techniques and functions like downloading or native menu items are not implemented yet.
This will cause developer teams, which want to add new functionalities to their applications to use frameworks that reduce their workload instead of increasing it.
Especially since most of the libraries used at web-frameworks are in terms of compensating a lack of features~\cite{explorationstudy}.
Since Electron is used by many well-known companies like Microsoft with their Teams or Visual Studio applications,
this impacts the interests of improving the framework as well as the popularity for small development teams.


Summarized, Tauri improves some problems Electron has to deal with due to its architecture and underlying frontend and backend parts.
Nevertheless, it is not a suitable option for either implementing applications that are aimed to be provided to millions of users, or small development teams that do not have the resources required by Tauri applications.
As it was pointed out by the authors of~\cite{explorationstudy} development teams using Electron have a median number of 1 core developer.
Using Tauri will result at increasing complexity for those core teams, depending on the knowledge of system-related programming principles, although the background of the developers was not determined by the authors.
Thus, the release time of applications or new features at least at the beginning of projects will shift further.
This could affect developers or even companies that want to obtain a balanced mix between release intervals and invested resources.
Tauri is a complex framework compared to other common cross-platform web-frameworks for implementing desktop applications.
However, it provides the possibility for developing efficient and small-sized applications compared to Electron, although they are currently limited by the Rust core.


\subsection{Further work}
Since this paper is based one the first stable release of Tauri, the topic has to be observed in the future, especially if the developers are improving their framework and documentation considering the requirements of the community.
This could result in further performance comparisons between the frameworks or the productive operation of applications build with Tauri, especially with applications requiring intensive cpu workload or complex calculations.
It still needs further research or observation of how the Electron developers are dealing with the issues, the Tauri developers have mentioned in the future.
This includes investigation of developers using Electron if they want to migrate to Tauri or respectively as well as the reasons for that migration.
Since Tauri is a framework with a new approach of using a system-related programming language for developing cross-platform desktop applications, further investigation is necessary in terms of how this affects establishing Rust, especially as web-development language.
Although the resource consumption of Electron is often criticized, there is no scientific publication what influence this has to the consumers choice of using those applications, since modern desktop computers do not suffer from a lack of memory in general.
Furthermore, it could be investigated if or how this affects companies or development teams in their decision to migrate from Electron to Tauri.



% macht speicher so einen großen einfluss