%! Author = Kevin
%! Date = 29.07.2022

% Preamble
\documentclass{acm_proc_article-sp}

% Packages
\usepackage{graphicx}
\graphicspath{images/}
\usepackage{epstopdf}
\usepackage{xr-hyper}
\usepackage{hyperref}
\usepackage{todonotes}
\usepackage[nodayofweek]{datetime}
\usepackage{acronym}
\usepackage{subfiles}

% Commands
\newdate{dateAtomShell}{15}{07}{2013}
\newdate{dateRenameElectron}{23}{04}{2015}
\date{\displaydate{date}}
\pagenumbering{arabic}
\newcommand{\secref}[1]{\autoref{#1}.~\nameref{#1}}

% Document
\begin{document}

    \title{{\ttlit Electron} and {\ttlit Tauri}: Comparison of web-frameworks for building cross-platform desktop applications}
    \numberofauthors{1}
    \author{
        \alignauthor
        Kevin L\"uttge \\
        %\titlenote{Dr.~Trovato insisted his name be first.}\\
        \affaddr{University of Kassel}\\
        %\affaddr{Software Engineering Research Group}\\
        %\affaddr{Department of Computer Science and Electrical Engineering}\\
        %\affaddr{Wilhelmsh\"oher Allee 73, 34121 Kassel, Germany}\\
        %\email{trovato@corporation.com}
    }

    \maketitle

    \begin{abstract}
        Web-frameworks like Electron provide the possibility to easily develop cross-platform desktop applications by using common web-technologies
        such as \ac{HTML}, \ac{CSS} and JavaScript.
        The increasing popularity of those frameworks is shown by various day-to-day applications namely WhatsApp, Discord or Visual Studio Code
        that are built with Electron.
        Two years ago Tauri came up, whose developers claimed it to be faster, smaller and more focused on security than other alternative frameworks.
        This paper will discuss the differences and similarities between Electron and Tauri, in terms like architecture, performance or security.
        Therefore a counter application is implemented for each framework to analyze and compare the development process.
    \end{abstract}

    \section{Introduction}\label{sec:introduction}
Cross-platform web-frameworks like Electron, Tauri or Flutter provide the possibility of using standard web-technologies like \ac{HTML}, \ac{CSS} and JavaScript or entire web-frameworks
like Angular to develop or migrate classic desktop applications to web applications.
The traditional approach of implementing desktop applications needs the developer to consider different \ac{API} or environments of the major  \ac{OS}s
Windows, Mac and Linux.
In fact, each application has to be implemented multiple times to adapt \ac{OS}-specific requirements and produces redundant code.
In context of consumer applications, the usage of frameworks like Electron allows companies and product owners to develop new target groups.
But it also increases the pool of potential applicants, since they provide web developers technologies and features to implement desktop applications without the knowledge of standard programming languages used for those kind of applications like C/C++ or Java.
Furthermore, institutions or universities and their students benefit from it, because applications can be used by many devices, due to the fact of cross-platform capabilities and decrease costs
since they do not need particular hardware to be executed.

\subsection{Background}\label{subsec:background}
In recent years web-frameworks for building desktop applications have experienced more and more attention, which can be expressed by the number of articles related to such topics or the temporal progression of Google Trends related to this topic.
This is owed to the fact of fundamental advances at web technologies like TypeScript, as an improvement of JavaScript, or frontend frameworks like Angular or React~\cite{pernice:icalepcs2019-wempr006},
which led to a wider use of such technologies for various scenarios.
A lot of every day applications used by computer scientists like Visual Studio Code or GitHub Desktop, but also applications with usage spread over industry sectors
like Microsoft Teams, Skype or Discord and even Social Media applications like WhatsApp or Twitch are implemented utilizing cross-platform web-frameworks.
The trend of using those applications has even grown up since the outbreak of SARS-CoV-2 in early 2020~\cite{Gorbalenya2020} and the enlarged number of employees working from home
as a result of lockdowns all over the world.
This led to an increasing number of cross-platform web-frameworks with different approaches in context of building, security or performance to provide a fast and simple way even for small development teams
to use web-technologies for implementing desktop applications, since their cross-platform capabilities prevent redundant implementation of the same application for different \ac{OS}s.

%Research Questions%
\subsection{Motivation}\label{subsec:motivation}
As mentioned in Chapter~\ref{subsec:background}, various cross-platform web-frameworks came up using different technologies or languages for their backend and frontend core.
One of the oldest and most commonly used frameworks is Electron, which is backed up by several popular applications mentioned before utilizing this framework.
Therefore, Electron has become a standard in context of cross-platform development of desktop applications and almost every new framework invented is benchmarked against it like Flutter~\cite{flutter}, JavaFX~\cite{electron-javafx} or Electron~\cite{electron-nwjs}.
Two years ago a new framework called Tauri was introduced, based on the discontent with Electrons high memory consumption or insecurity by its developers.
It was designed to improve several aspects of Electron, especially in case of performance, memory usage and security~\cite{tauri}.
Since these statements are made by the developers and thus may be subjective, this paper aims to provide an objective overview to the reader of both frameworks, Electron and Tauri.
Therefore, fundamental architecture as well as the frontend and backend core of each framework is explained in detail at Chapter~\ref{sec:electron} for Electron and at Chapter~\ref{sec:tauri} for Tauri.
This technological knowledge is applied by implementing a basic application which will be described in detail at Chapter~\ref{sec:implementation} and analyzes different aspects of both.
To obtain an objective comparison between both frameworks, Chapter~\ref{sec:summary} examines the results of previous Chapters
to isolate advantages and disadvantages to the reader.
At the end of this paper, the comparison results of Chapter ~\ref{sec:summary} are contrasted against the statement of the Tauri developers from~\cite{tauri} and being discussed at Chapter~\ref{sec:conclusion} to provide an unbiased
evaluation of both frameworks.

\subsection{Related Work}\label{subsec:related-work}
An exploratory study of the utilization of different frameworks for desktop apps has worked out that applications developed with cross-platform web-frameworks are made for various kinds of usage~\cite{explorationstudy}.
The authors also empirically documented that those frameworks are mostly used by developer teams with a median size of 1, which is a direct result of the
amount resources classical desktop development consumes.
Nevertheless, they also discovered disadvantages using such web-frameworks, like a high number of used libraries.
This is due to the fact of compensating a lack of features provided by the frameworks.
But also a high ratio of platform-related issues to all issues of 20\%.

The increasing popularity of web-technologies has been shown by~\cite{pernice:icalepcs2019-wempr006}.
Since they identified their native Java Swing desktop application as a bottleneck, the authors decided to replace it with a technology stack providing long term sustainability.
Therefore, they used several web-technologies like AngularJS or Typescript to implement ``a web-based tool for configuring experiments on the National Ignition Factory``~\cite{pernice:icalepcs2019-wempr006}.
But they also came in touch with typical problems of the JavaScript ecosystem like exit or replacement of widely used technologies forcing them to migrate from
AngularJS to its successor Angular.
It has to be mentioned that the authors emphasized the community support especially, in case of migration by providing tools for those cases.

Using Electron as a standard benchmark for cross-platform web-frameworks is corroborated by various papers comparing it to other frameworks like JavaFX~\cite{electron-javafx}, NW.js~\cite{electron-nwjs} or Flutter~\cite{flutter}.
Both authors compared different web-frameworks against Electron.
Although the benchmark of the authors of~\cite{electron-javafx} was made with different \ac{IDE}s and the comparison between Flutter and Electron by~\cite{flutter} was based on a beta version of Flutter,
both pointed out that at the one hand Electron has better performance in case of execution time and CPU consumption and on the other hand it provides more features than the compared frameworks.


    \section{Electron}
\label{sec:electron}
Electron originally was released as Atom Shell by GitHub at \displaydate{dateAtomShell}~\cite{githubReleaseV010} and downloaded approx. $83\,000\,000$ times\footnote{According to \url{https://npm-stat.com/charts.html?package=electron&from=2013-07-13&to=2022-07-30}}
The intention of the developers was ``to handle the Chromium/Node.js event loop integration and native APIs``~\cite{sawicki_2015} for the Atom Editor.
On \displaydate{dateRenameElectron} it was renamed to Electron and announced that the developers want to provide a framework that allows to build desktop applications with the using web-technologies.
Many day-to-day apps have been implemented using Electron like Discord, Twitch or even Microsoft Teams.
This has lead to an increasing community which can be expressed numerically based on GitHub Statistics~\cite{GithubElectron} \\ \\:
\begin{tabular} {| c | c | c | c | c |}
    \label{tab:electron:statistics}
    Stars      & Forks     & Watching & Used by    & Contributors \\ \hline
    $130\,000$ & $13\,700$ & $2\,900$ & $244\,568$ & $1\,126$
\end{tabular}

\subsection{Architecture}
\label{subsec:electron:architecture}
Electrons architecture fundamentally relies on Chromium, an open-source browser which is included in each Electron executable and has a lot common with it.
As Chromium Electron is based on a multi-process model, containing of a single process called \textbf{main} and several processes, one for each window, called \textbf{renderer}
\begin{figure}[ht]
    \label{fig:electron:model}
    \centering
    \includegraphics[width=0.4\textwidth]{images/electron-model}
    \caption[Bla]{Multi-process Model Electron from~\cite[Fig. 1.7]{electron-in-action}}
\end{figure}
\begin{description}
    \item[\textbf{Main}:] \hfill \\ This is the main entry point of the application and responsible for lifecycle management like starting or quitting the app.
    The \texttt{main} process uses so called \texttt{BrowserWindow} module to create and manage each \texttt{renderer} process which is loading web pages into it.
    Since just the \texttt{main} process is running inside a NodeJS environment, it is the only part of the application that can import NodeJS modules using \texttt{require}.
    This forces each \texttt{renderer} process to interact with the \texttt{main} process if they want to consume system \ac{API}s for purposes like saving files or opening dialogs.
    \item[\textbf{Renderer}:] \hfill \\ As it is implied by the name a \texttt{renderer} process renders web content by loading web pages into it and presenting them to the user.
    Additionally, javascript code can be loaded and executed inside a process.
    Each \texttt{renderer} process can be created or destroyed by the main process using the \texttt{BrowserWindow} module as mentioned before.
    This leads to the fact, that \texttt{renderer} processes are isolated from each other following the Chromium principles of a multi-process model and is reasoned by limited affection of
    faulty or malicious code on the entire app.
    The \texttt{renderer} processes are only able to communicate between each other indirectly via the \texttt{main} process.
    This is called \ac{IPC}
\end{description}

\subsubsection{IPC}
As mentioned above \texttt{main} and \texttt{main} processes are only able to communicate using \ac{IPC}.
Therefor Electron provides two modules, one for the \texttt{main} process, called \texttt{ipcMain} and one for \texttt{renderer} process, called \texttt{ipcRender}.
Both of them are Node.JS \texttt{eventEmitter} modules and capable of executing asynchronously and synchronously communication either uni- or bidirectional.

\subsubsection{Context Isolation}
Electrons multi-process model also intends distinct purposes for each process.
As described before only the \texttt{main} process has access to node modules.
In contrast to that only the \texttt{renderer} process has access to \ac{HTML} \ac{DOM}s.
Since using just \ac{IPC} represents a major security issue, Electron introduced a feature called \textbf{Context Isolation}.
This splits the logic of a \texttt{renderer} process into two different contexts on the one hand the \texttt{renderer} process as already described and on the other a so called \texttt{preload script}.
The \texttt{preload script} is attached to the \texttt{main} process at the creation of the \texttt{BrowserWindow} module and at its context has access to both node modules and \ac{HTML} \ac{DOM}s.
Although the \texttt{preload script} and the \texttt{renderer} process do both own a window object which provides the displayed browser window it is not the same object since they are both running at
different contexts.
The \texttt{preload script} consists of a \texttt{contextBridge} module which is responsible for safely exposing selected properties of the \texttt{main} process to the \texttt{renderer} and vice versa.
Inside this module an \ac{API} can be defined for providing access with \ac{IPC} objects to resources of different processes.

%each view is rendered at a separate process, called \textbf{renderer}.
%The reason for this is that faulty or malicious code is limited in its affection at the whole app.
%Each process is controlled by a single process, which is also the entry point of the app, called \textbf{main}.

\subsection{Frontend}
\label{subsec:electron:frontend}
This subsection deals with the front-end Electron is using, which other frontend-frameworks are supported, etc.

\subsection{Backend}
\label{subsec:electron:backend}
At this subsection we will take a look at the Node JS Backend of Electron and analyse the advantages and disadvantages of NodeJS
    %\section{Tauri}
\label{sec:tauri}
History, why/when it was released, written in ..., general information

\subsection{Architecture}
\label{subsec:tauri:architecture}
Same as described for Tauri.Ref to introduction\ref{subsec:electron:arch}

\subsection{Frontend}
\label{subsec:tauri:frontend}

\subsection{Backend}
\label{subsec:tauri:backend}



    %\section{Implementation of a counter \\ application}
\label{sec:implementation}
As already mentioned at~\ref{subsec:motivation} this section describes the underlying work for this paper and will guide through the entire development process of Electron and Desktop applications.
To obtain comparable results, the methodology of comparison will be explained shortly.
For this paper a basic counter application has been implemented in both Electron and Tauri using just bare \ac{HTML},\ac{CSS} and JavaScript, although both frameworks are supporting most common frontend frameworks like Angular, React or Vue.js.
This decision was made because the paper focuses on presenting the differences and similarities of each framework to the reader and thus using complete frameworks will both blow up the entire application and not concentrate on the essentials.
The counter application consists of a simple Counter which is displayed and two buttons providing the possibility to increment or decrement the counter.
Both buttons will have an event listener, that sends \ac{IPC} messages to the backend and as response get the new calculated counter value which then will be displayed to show the entire communication chain of each framework.
Therefore, each application will contain the same \ac{HTML} content as it can be seen in \figref, that is displayed and only the API calls of the frameworks will differ, depending on their architecture.
%This section will include a short explanation of the counter app as an example application.
%It will show the features and define the requirements for an objective comparison.

\subsection{Method}
\label{subsec:method}
For benchmarking the applications in case of build time GitHub Actions are used.
Thus, each project has its own GitHub repository with a workflow.yaml defining the actions' workflow.
Each Action will run on the three major operating systems macOS, Windows and Linux, set the prerequisites for the framework respectively and use the recommended build tool, \texttt{electron-forge} for Electron and the \texttt{tauri build} command for Tauri.
This will result in three jobs for each project, whereas the build time for each framework on each os will be measured, since usual prerequisites are installed once and thus not measured.
Time of execution is measured using hyperfine\footnote{\url{https://github.com/sharkdp/hyperfine}} which is a command line benchmark tool.
Therefore, the executable of each framework will be started by command line, with hyperfine switched in front.
To see the difference between a cached and uncached execution, the first run does not have any warmup actions and so prevent the operating system from loading the application into the filesystem cache.
This results in following command \shellcmd{hyperfine --runs 10 --warmup 2 'start <executable>}
To gather meaningful data, hyperfine will run 10 instances of the executable to obtain a min-max range as well as a mean execution time.

For memory consumption the python library memory-profiler\footnote{\url{https://pypi.org/project/memory-profiler/}} will measure the memory consumption of each executable over a timespan of 60 seconds,
enough to get from startup to idle state.
This timespan will be set to 60 seconds, to allow interaction with the application as well as complete startup and idle state.
Since the applications rely on multiprocess models and may also spawn children, these are recorded too, to gather trustful memory consumption measurements.
To archive this measurement following memory profiler command is used \shellcmd{mprof run --include-children --multiprocess --timeout 60 <executable>}
It is important to notice that Tauri comes with a single executable file, whereas Electron ships with an installer which has to be executed first in order to get the actual executable application running.
All Electron measurements will use this installed executable as foundation.
%This subection section will take the findings of~\ref{sec:electron} into action and analyze the development and building process as well as the performance of the executable application.

\subsection{Development}
\label{subsec:impl:dev}
%As mentioned at~\ref{subsec:electron:architecture} the main process of Electron is running inside the Node.js environment, meaning that this is the only place where Node-Modules can be \textbf{required} and used.
%At this subsubsection the general development process of an Electron application is discussed but also the actual development using the screencast
%application as an example.
%Are there any templates provided, which dependencies and tools are needed for implemention, debbuging and testing.
%Are they already provided by the installer?
%Also have a short look at the documentation e.g.are there any guides, community feedback.

\subsection{Build}
\label{subsec:impl:build}
%This subsubsection deals with the whole building process of the example Electron app e.g.\ Package manager, cross-platform support, build tools, build time etc.

\label{tab:buildtime:statistics}

\subsection{Performance}
\label{subsec:impl:performance}
\subsubsection{Build Time}
\label{subsubsec:perf:buildtime}
Lore ipsum
\\
\begin{tabular} {| c | c | c | c |}

    \hline
    \multicolumn{4}{|c|}{Building time in seconds} \\ \hline
    Framework & \multicolumn{3}{|c|}{Operating System} \\ \hline
    & Windows & Ubuntu & MacOS \\ \hline
    Tauri & 863 & 616 & 389 \\ \hline
    Electron & 115 & 121 & 44 \\ \hline

\end{tabular} \\ \\
\subsubsection{Memory Consumption}
\label{subsubsec:perf:memory}

Lore ipsum \\
\begin{figure}[ht]
    \centering
    \includegraphics[width=0.5\textwidth]{images/ElectronMemCons.jpeg}
    \caption[Bla]{Memory consumption of Electron executable obtained from memory profiler}
    \label{fig:electron:memory}
\end{figure}
\subsubsection{Execution Time}

\begin{figure}[ht]
    \centering
    \includegraphics[width=0.5\textwidth]{images/TauriMemCons.jpeg}
    \caption[Bla]{Memory consumption of Tauri executable obtained from memory profiler}
    \label{fig:tauri:memory}
\end{figure}
\subsubsection{Execution Time}
\label{subsubsec:perf:execution}

\begin{tabular} {| c | c | c | c |}

    \hline
    \multicolumn{4}{|c|}{Execution time of Electron [ms]} \\ \hline
     \multicolumn{4}{|c|}{}\\ \hline
     & Mean with Standard Deviation & Min & Max     \\ \hline
    Without caching & 17.3 $\pm$ 23.3 & 6.5 & 82.9  \\ \hline
    With caching & 18.0 $\pm$ 14.8 & 7.5 & 49.6 \\ \hline

\end{tabular} \\ \\


\begin{tabular} {| c | c | c | c |}

    \hline
    \multicolumn{4}{|c|}{Execution time of Tauri [ms]} \\ \hline
    \multicolumn{4}{|c|}{}\\ \hline
    & Mean with Standard Deviation & Min & Max     \\ \hline
    Without caching & 17.1 $\pm$ 10.5 & 5.8 & 35.4  \\ \hline
    With caching & 15.1 $\pm$ 7.9 & 5.2 & 29.0 \\ \hline

\end{tabular} \\ \\
%At this subsubsection the actual performance of the example application is analyzed.
%Memory consumption, known security issues, bugs, freezes etc.
%This subsection section will take the findings of~\ref{sec:tauri} into action and analyze
%the development and building process as well as the performance of the executable application.

    %\section{Summary}
\label{sec:summary}
%This section will take the knowledge of \ref{sec:electron} and \ref{sec:tauri} as well as the analysis of \ref{sec:implementation} to contrast them.
To summarize the content of chapters~\ref{sec:electron} and~\ref{sec:tauri} as well as the measurement results of section~\ref{sec:implementation}
both frameworks will be put in contrast to point out differences or similarities of each framework
\subsection{Differences}\label{subsec:differences}
As described in sections~\ref{sec:electron} and~\ref{sec:tauri} both frameworks rely on different technologies to build up their backend or to render the content of a web page.
Electron uses the Chromium web browser which is also shipped with the compiled executable resulting at higher binary size than Tauri.
Tauri relies on the underlying operating systems web engine and outsources the rendering to that engine to save memory.
But this causes developers to take care of the different operating systems and their specific requirements, whereas Electron solves this problem with the Chromium web browser.
Although both frameworks provide cross-platform capabilities and support the major operating systems, only Electron provides full cross-platform capability meaning that developers are able to implement an application only once
and run it on every operating systems without differing visual presentations.
But it also has to be mentioned that the Electron executable ships necessary libraries as extern parts, whereas the Tauri build tools are packaging required dependencies and libraries into a single executable file.
This results at higher security since only a successful decompilation opens gateways to potential attackers, whereas Electron libraries could be compromised without decompilation and thus corrupt the executable.
Another important subject that both frameworks differ is the used language.
Electron applications can be written completely using JavaScript providing easy access to web-developer whereas development using Tauri requires knowledge with Rust, which has a high learning curve due to its complexity compared to JavaScript.
Nevertheless, chapter~\ref{subsubsec:perf:execution} points out that the performance of Tauri applications is slightly better in case of base applications.
The reader has to notice, that all measurements were done using a simple counter application without any cpu or memory intense workloads and thus the better performance of the Rust backend compared to JavaScript might have a bigger impact on execution time.
Another difference of both frameworks is at the subject of build time since Rust, in spite of improvements has a high compile complexity~\cite{rustCompileTime} although this is mostly avoided by caching and may not have such a big impact
at day-to-day development.
The Rust backend itself points out another difference between Tauri and Electron since it guarantees memory and thread safety and its model of ownership forces the developer to consider the efficiency of his/her application~\cite{klabnik2019rust}.
At points of library support Rusts own package manager cargo is limited in terms of the amount of provided libraries compared to \ac{NPM}.
This results in the fact that some functionalities are not provided by libraries and thus have to be implemented by the developers on their own.
But also security aspects differ between both frameworks, although both are concentrating the main access to os-specific operations at the main or core process.
Especially Tauris Isolation Pattern has to be pointed out, since it allows to execute suspicious frontend calls at an isolated, trusted sandbox environment before it is forwarded to the main core.
This helps to avoid malicious software getting access to the core or obtaining privileges it should not have.
Since Tauri applications are running serverless there is no possibility for potential attackers sniffing the network traffic between the frontend and the backend like it is done by Electron, but also each communication between Tauris backend and
frontend is encrypted providing additional complexity to monitor the data that is exchanged~\cite{tauri}.
Another security aspect both frameworks differ are their dependencies.
Since Electron uses~\ac{NPM} for its dependencies, it is up to the developer to update them and therefore is a possible open gate for potential attacks like the zero-day exploit of Log4J~\cite{bsi}.
Although this could also happen to cargo the features of Rust itself prevent most of the possible attacking space that may occur at \ac{NPM} packages.


%This subsection will work out the differences between the two frameworks
%with regard to Architecture, Frontend, Backend, Development, Build and Performance.

\subsection{Similarities}\label{subsec:similarities}
Although both frameworks differ at various points especially in case of security and performance they also share similarities.
Both are supporting most common and modern web-frameworks like React or Angular by providing boilerplate generation tools that create basic projects with a ready-to-use environment where the frameworks are set up to work with Tauri respectively Electron.
Electron as well as Tauri use the multiprocess model as their fundamental architecture, whereas, as pointed out in chapter~\ref{subsec:electron:architecture} and~\ref{subsec:tauri:architecture} a main process is used as entry point
to the application, which is able to create and manage new browser window processes or enables communication capabilities or system-specific access.
At least for basic applications they also share a just slightly different execution time.
Both frameworks allow developers to decide which parts of their \ac{API} are exposed to the different processes, although this is done in different ways, Electron with its preload script and the contextBridge and Tauri by simply prevent each part of the API that is not
used being packaged.
Despite the fact both frameworks use slightly different implementations for their \ac{IPC}, Tauri implementing a protocol similar to \ac{JSON-RPC} and Electron is implementing the \ac{HTML} standard \ac{SCA}~\cite{ElectronDoc},
the communication always passes the main process either directly or indirectly as message broker.
But unlike Tauri, Electron provides the possibility to invoke two-way \ac{IPC} request from the main process to the renderer processes, whereas Tauri allows commands only be emitted by the frontend, limiting the impact
of a corrupted main process.


%This subsection will work out possible similarities between the two frameworks with regard to Architecture, Frontend, Backend, Development, Build and Performance.
    %\section{Conclusion}
\label{sec:conclusion}
%At this section a conclusion is made based on the advantages/disadvantages of each framework
%described in \ref{sec:conclusions}.
%Are there different use-cases for each framework, are they concurrent to each other, etc.?
Reminding the intention of the developers of Tauri mentioned at section~\ref{sec:tauri} and contrasting them to the measurements discussed in section~\ref{sec:implementation}
it can be determined that some aspects like execution time or security of Tauri applications compared to Electron are accurate.
Tauri applications have a more efficient performance in aspects of memory usage or execution time, due to the fact of using Rust as backend core.
Nevertheless, some benchmarks provided by Tauri, especially in case of memory consumption could not be reproduced or investigated further since those are not provided public.
But as mentioned in section~\ref{sec:summary} Tauri compared to Electron does a step towards the traditional web development, where system-specific requirements have to be considered and
therefor at worst case an application for each operating system has to be implemented.
Although this might need only small changes on the entire code base redundant code follows from this.
Tauri has experienced much attention since its first release but at the current state is not suitable for big applications in production mode.
This statement is based on different aspects.
First, the documentation is difficult to understand resulting in paraphrases or one-sentenced explanations of entire architecture aspects like their design patterns, than actual explaining them.
This results in uncertainties how the Tauri framework is doing tasks in detail.
Especially the security section is rather describing possible threats that can occur during development than explaining their implemented solutions or how they affect existing code.
Although documentations do not have to be objective, the Tauri documentation often compares parts of it against Electron applications even if some of them are referencing old and deprecated modules like ASAR files~\cite{tauri}. \\
Second, the knowledge of Rust required to implement more complex applications that need access to the Rust core.
Electron can be written by just using JavaScript and thus lowers the bounds for small web-development teams, since JavaScript is widely used at this branch.
For those teams, implementing a Tauri application will result in learning a complete new programming language and in worst case if they do not have any knowledge in comparable languages like C++
learning complete new programming patterns like \ac{OOP} principles or memory management.
This consumes more resources during the development process and thus will lead to an increasing timespan between the initial development and a release of the application.
Although Tauri has claimed to face this issue and announced to provide a backend that can be written using different languages, it is neither implemented nor scheduled yet.
Several standard web techniques and functions like downloading or native menu items are not implemented too.
This will cause developer teams add new functionalities to their applications since they are not provided by Electron yet.
Third the community itself as well as the support of the developers, since Tauri and especially Rust suffers from a lack of functionalities and Features that are provided Node.js or Electron.
Since Electron is used by many well-known companies like Microsoft with their Teams or Visual Studio applications or Discord and their same named application,
this impacts the interests of improving the framework as well as the popularity for small development teams.

Summarized Tauri improves some problems Electron has to deal with due to its architecture and underlying frontend and backend parts but is not suitable option for implementing applications
that are aimed to be provided to millions of users or small development teams that do not have the resources that are required to implement Tauri applications.
As it was pointed out by the authors of~\cite{explorationstudy} development teams using Electron have median number of 1 core developer.
Using Tauri will result increasing complexity for those core teams although the background of the developers was not determined by the authors depending on the knowledge of system-related programming principles.
Thus, the release time of applications or new features at least at the beginning of projects will shift further.
This could affect developers that or even companies that want to obtain a balanced mix between release intervals and invested resources.
It is a complex framework compared to other common cross-platform web-frameworks for implementing desktop applications but provides the possibility to develop efficient and small-sized applications compared to Electron, although
they are currently limited by the Rust core.


\subsection{Further work}
Since this paper is based one the first stable release of Tauri, the topic has to be observed in the future, especially if the developers are improving their framework and documentation considering the requirements of the community.
This could result in further performance comparisons between both frameworks or the productive operation of applications build with Tauri, especially with applications requiring intensive cpu workload or complex calculations.
It also needs further research or observation how the developers of Electron are dealing with the issues, the Tauri developers have mentioned in the future, including investigation of development that are using Electron if
they want to migrate to Tauri or respectively as well as the reasons for that.
Since Tauri is a framework with a new approach of using a system-related programming language of developing cross-platform desktop applications, further investigation in terms of how this affects establishing Rust especially as web-development language is necessary.
Although the resource consumption of Electron is often criticized, there is no scientific publication how this influences the consumer choice of using those applications, since modern desktop computers do not suffer from a lack of memory in general
or how this affects companies in their decision to migrate from Electron to Tauri.



% macht speicher so einen großen einfluss
    \section{List of Abbreviations}
\label{sec:list-of-abbreviations}

\begin{acronym}
    \acro{HTML}[HTML]{Hypertext Markup Language}
\end{acronym}

\begin{acronym}
    \acro{CSS}[CSS]{Cascading Style Sheets}
\end{acronym}

\begin{acronym}
    \acro{API}[API]{Application Programming Interface}
\end{acronym}

\begin{acronym}
    \acro{IPC}[IPC]{Inter-Process Communication}
\end{acronym}

\begin{acronym}
    \acro{OS}[OS]{Operating System}
\end{acronym}


\begin{acronym}
    \acro{DOM}[DOM]{Document Object Module}
\end{acronym}

\begin{acronym}
    \acro{IDE}[IDE]{Integrated Development Environment}
\end{acronym}


    \newpage

    \bibliography{main}
    \bibliographystyle{plain}

    \balancecolumns

\end{document}
